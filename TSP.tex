%%%%%%%%%%%%%%%%%%%%%%%%%%%%%%%%%%%%%%%%%
% Beamer Presentation
% LaTeX Template
% Version 1.0 (10/11/12)
%
% This template has been downloaded from:
% http://www.LaTeXTemplates.com
%
% License:
% CC BY-NC-SA 3.0 (http://creativecommons.org/licenses/by-nc-sa/3.0/)
%
%%%%%%%%%%%%%%%%%%%%%%%%%%%%%%%%%%%%%%%%%

%----------------------------------------------------------------------------------------
%	PACKAGES AND THEMES
%----------------------------------------------------------------------------------------

\documentclass{beamer}
\usepackage[OT4]{polski}
\usepackage[utf8]{inputenc}

\mode<presentation> {

% The Beamer class comes with a number of default slide themes
% which change the colors and layouts of slides. Below this is a list
% of all the themes, uncomment each in turn to see what they look like.

%\usetheme{default}
%\usetheme{AnnArbor}
%\usetheme{Antibes}
%\usetheme{Bergen}
%\usetheme{Berkeley}
%\usetheme{Berlin}
%\usetheme{Boadilla}
%\usetheme{CambridgeUS}
%\usetheme{Copenhagen}
%\usetheme{Darmstadt}
%\usetheme{Dresden}
%\usetheme{Frankfurt}
%\usetheme{Goettingen}
%\usetheme{Hannover}
%\usetheme{Ilmenau}
%\usetheme{JuanLesPins}
%\usetheme{Luebeck}
\usetheme{Madrid}
%\usetheme{Malmoe}
%\usetheme{Marburg}
%\usetheme{Montpellier}
%\usetheme{PaloAlto}
%\usetheme{Pittsburgh}
%\usetheme{Rochester}
%\usetheme{Singapore}
%\usetheme{Szeged}
%\usetheme{Warsaw}

% As well as themes, the Beamer class has a number of color themes
% for any slide theme. Uncomment each of these in turn to see how it
% changes the colors of your current slide theme.

%\usecolortheme{albatross}
%\usecolortheme{beaver}
%\usecolortheme{beetle}
%\usecolortheme{crane}
%\usecolortheme{dolphin}
%\usecolortheme{dove}
%\usecolortheme{fly}
%\usecolortheme{lily}
%\usecolortheme{orchid}
%\usecolortheme{rose}
%\usecolortheme{seagull}
%\usecolortheme{seahorse}
%\usecolortheme{whale}
%\usecolortheme{wolverine}

%\setbeamertemplate{footline} % To remove the footer line in all slides uncomment this line
%\setbeamertemplate{footline}[page number] % To replace the footer line in all slides with a simple slide count uncomment this line

%\setbeamertemplate{navigation symbols}{} % To remove the navigation symbols from the bottom of all slides uncomment this line
}

\usepackage{graphicx} % Allows including images
\usepackage{booktabs} % Allows the use of \toprule, \midrule and \bottomrule in tables

%----------------------------------------------------------------------------------------
%	TITLE PAGE
%----------------------------------------------------------------------------------------

\title[TSP ACU]{Problem komiwojażera} % The short title appears at the bottom of every slide, the full title is only on the title page

\author{Miriam Końska, Michał Wierzbicki} % Your name
\institute[IIUWr] % Your institution as it will appear on the bottom of every slide, may be shorthand to save space
{
University of Wrocław \\ % Your institution for the title page
\medskip
% \textit{john@smith.com} % Your email address
}
\date{\today} % Date, can be changed to a custom date

\begin{document}

\begin{frame}
\titlepage % Print the title page as the first slide
\end{frame}

\begin{frame}
\frametitle{Spis treści} % Table of contents slide, comment this block out to remove it
\tableofcontents % Throughout your presentation, if you choose to use \section{} and \subsection{} commands, these will automatically be printed on this slide as an overview of your presentation
\end{frame}

%----------------------------------------------------------------------------------------
%	PRESENTATION SLIDES
%----------------------------------------------------------------------------------------

%------------------------------------------------
\section{Problem komiwojażera} % Sections can be created in order to organize your presentation into discrete blocks, all sections and subsections are automatically printed in the table of contents as an overview of the talk
%------------------------------------------------

\subsection{Zarys problemu} % A subsection can be created just before a set of slides with a common theme to further break down your presentation into chunks


\begin{frame}
\frametitle{Zarys problemu}
Problemem komiwojażera nazywamy znalezienie cyklu Hamiltona o najmniejszej karze sumarycznej w grafie pełnym.
Cykl Hamiltona jest zamkniętą ścieżką w grafie przechodzącą dokładnie raz przez każdy wierzchołek, z wyjątkiem wierzchołka początkowego(końcowego).

\end{frame}


\subsection{Cykl Hamiltona}

%------------------------------------------------
\begin{frame}
\frametitle{Cykl Hamiltona}
Znalezienie cyklu Hamiltona w grafie jest problemem należącym do klasy problemów NP.
Oznacza to, że nie istnieje wielomianowy algorytm (względem ilości wierzchołków) rozwiązujący ten problem. Oznacza to, że dla dużych danych wejściowych nie jesteśmy w stanie obliczyć odpowiedzi w sensownym czasie.
\end{frame}

\subsection{Model matematyczny}
\begin{frame}
\frametitle{Model matematyczny}
Dane wejściowe: \\
Graf pełny $G=(V,E)$  \\
Macierz odległości $D_{nxn}$ \\
Funkcja kary $ f : (E,K) \rightarrow {\rm I\!R} $ , gdzie $E$ to zbiór wierzchołków $G$ a $K$ to czasy najpóźniejszego przybycia, po którym naliczana jest kara.

Dane wyjściowe: \\
Cykl $C$ taki, że suma kar za przekroczenie terminu jest minimalna.
\end{frame}





%------------------------------------------------
\section{Algorytm mrówkowy}
%------------------------------------------------

\subsection{Idea algorytmu}

\begin{frame}
\frametitle{Algorytm mrówkowy}
\begin{itemize}

\item Każda mrówka po danej iteracji zostawia na krawędziach feromon o wartości f \\
\item Z funckją feromonu jest związany współczynnik $\alpha$, który mówi nam jak bardzo mrówki będą w swoich wyborach uwzględniać wartość feromonu na krawędziach. \\
\item Funkcja ruletki jest wykorzystywana do wyboru kolejnego miasta do odwiedzenia na trasie.\\
\item Współczynnik parowania - stała z przedziału $(0,1)$ \\
\item Minimalna ilość feromonu na krawędzi - zabezpieczenie przed utratą danych o danej krawędzi \\
\item Liczba mrówek - wybierana na początku \\
\end{itemize}
\end{frame}
%----------------------------------------------------------------------------------------

\subsection{Pseudokod}

\begin{frame}
\frametitle{Pseudokod}
1. Dla każdej mrówki wylosuj miasto początkowe \\
2. Niech każda mrówka przejdzie cykl Hamiltona w grafie: startując z miasta początkowego, przejdź do dotychczas nieodwiedzonego miasta posługując się funkcją rulteki, jednocześnie pamiętając dotychczas przebytą trasę. Gdy wszystkie miasta zostały odwiedzone powróć do punktu startowego.
3. Po zakończeniu trasy przez wszystkie mrówki dodaj na krawędzie, przez które przeszły wartość równą odwrotności długości przebytej drogi. \\
4. Jeśli w tej iteracji któraś z mrówek znalazła lepszą ścieżkę, to zapamiętaj to rozwiązanie.\\
5. Feromony na krawędziach przemnóż przez współczynnik parowania.\\
6. Wykasuj pamięć dla każdej z mrówek i zacznij kolejną iterację.\\


\end{frame}


\end{document}